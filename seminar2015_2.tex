\documentclass[12pt,a4paper]{article}
\usepackage{graphicx}
\usepackage[applemac]{inputenc} %% european characters can be used (Mac OS)
% ------------------------------------------------
\usepackage[T1]{fontenc}   %% get hyphenation and accented letters right
\usepackage{mathptmx}      %% use fitting times fonts also in formulas
\pagestyle{empty}                %% no page numbers!
\usepackage[left=35mm, right=35mm, top=15mm, bottom=20mm, noheadfoot]{geometry}

% begin the document
\begin{document}
\thispagestyle{empty}

\title{\textbf{Co-occurence Relationships Between Insect Pest and Disease from Farmer's Field Survey Data Revealed by Network Analysis}}
\author{Sith Jaisong \\
Plant Disease Management Group, CESD, IRRI\\ Los Ba\~{n}os, Philippines\\
s.jaisong@irri.org}
\date{} % <--- leave date empty
\maketitle\thispagestyle{empty} %% <-- you need this for the first page
% introduction should have the objective or the problem or even telling the reader what you wanna say, and 

% I'd like you to follow the American Phytopathological Society guidelines for abstract submission

%Presentation Title
%Capitalize only the first letter of the first word and any proper nouns, (e.g., Effect of pesticides on recovery of Didymella bryoniae from cucurbit vines). The title is limited to 150 characters including spaces. (Approximately 30 word count.) Registered names and trademarks are not permitted in title.
%
%
%Abstract Text
%Read the technical requirements and view the sample abstract before submitting your abstract.
%
%The abstract must be in one paragraph.
%DO NOT include the title, author name(s), or author affiliations in the abstract text field.
%Copy the abstract and paste it into the submission form abstract text box under the Abstract Copy/Body field header.
%Or type text in to the abstract field.
%Use the abstract toolbar to add formatting (italics, superscripts, subscripts, Greek or math symbols), or use the start coding <i> and the stop coding </i>, if you prefer.
%If the symbol is not available, spell it out (e.g., theta).
%Character limit is 1,490 characters including spaces (Approximately 250 word count).
%

% Length is 1542 characters including spaces. Please shorten to 1490.
% Last sentence makes more sense, but you've still misspelled a word.
Pests are the blocks in the big wall of yield losses that plant protection specialists must scale due to the need to increase crop productivity. Surveys of farmers' fields are useful sources of data to help us determine the importance of pests and understand complex relationships within agroecosystems. Co-occurrence patterns, in the form of network analysis, can used to explore potential relationships between elements. Spearman's rank correlation-based network analysis was used to identify relationships between the incidence of insect pests and diseases from survey data collected in 450 farmers' fields in irrigated lowland rice growing areas in five Asian countries. Network models revealed relationships between incidences and severities of damages by insect pests and diseases, which showed both positive and negative relationships. The model illustrated that the important pests and diseases are brown planthopper, whorl maggot, bacterial leaf streak and brown spot. Moreover, network structures changed with different seasons (wet or dry season). In wet season, incidence of bacterial leaf streak, whorl maggot damage, silver shoot and the number of brown plant hoppers showed strong co-occurrence. While the incidence of narrow brown spot, number of brown planthopper, and white backed planthopper showed strong co-occurrence in dry season. This study provides a substantiated apprache to guide plant protection specialists in the construction and interpretation of co-occurrence networks from crop health survey survey datasets.
 
\end{document}
















