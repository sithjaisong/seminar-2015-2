\documentclass[12pt,a4paper]{article}
\usepackage{graphicx}
\usepackage[applemac]{inputenc} %% european characters can be used (Mac OS)
% ------------------------------------------------
\usepackage[T1]{fontenc}   %% get hyphenation and accented letters right
\usepackage{mathptmx}      %% use fitting times fonts also in formulas
\pagestyle{empty}                %% no page numbers!
\usepackage[left=35mm, right=35mm, top=15mm, bottom=20mm, noheadfoot]{geometry}

% begin the document
\begin{document}
\thispagestyle{empty}

\title{\textbf{Co-occurence relationships between Insect Pest and Disease from Farmer's Field Survey Data Revealed by Network Analysis}}
\author{Sith Jaisong \\
Plant Disease Management Group, CESD, IRRI\\ Los Ba\~{n}os, Philippines\\
s.jaisong@irri.org}
\date{} % <--- leave date empty
\maketitle\thispagestyle{empty} %% <-- you need this for the first page
% introduction should have the objective or the problem or even telling the reader what you wanna say, and 

% I'd like you to follow the American Phytopathological Society guidelines for abstract submission

%Presentation Title
%Capitalize only the first letter of the first word and any proper nouns, (e.g., Effect of pesticides on recovery of Didymella bryoniae from cucurbit vines). The title is limited to 150 characters including spaces. (Approximately 30 word count.) Registered names and trademarks are not permitted in title.
%
%Sith: I totally forget that you used to mention  the abstract should be following the role from American Phytopathology socity
%
%Abstract Text
%Read the technical requirements and view the sample abstract before submitting your abstract.
%
%The abstract must be in one paragraph.
%DO NOT include the title, author name(s), or author affiliations in the abstract text field.
%Copy the abstract and paste it into the submission form abstract text box under the Abstract Copy/Body field header.
%Or type text in to the abstract field.
%Use the abstract toolbar to add formatting (italics, superscripts, subscripts, Greek or math symbols), or use the start coding <i> and the stop coding </i>, if you prefer.
%If the symbol is not available, spell it out (e.g., theta).
%Character limit is 1,490 characters including spaces (Approximately 250 word count).
%

Pests are the blocks of the big wall that agronomists have to deal with because of the need to increase crop productivity. Survey data from farmers' fields are the useful sources of information to help us to determine the importance of pests and understand the complex interactions of the agroecosystem. Spearman's rank correlation-based network analysis was conducted to identify the co-occurrence correlations of incidence of insect pests, and diseases from survey data collected from the 450 farmers's fields in lowland rice growing areas across five countries including India, Indonesia, Philippines, Thailand and Vietnam from 2007 to 2010.  Network analysis revealed interactions among insect pests and diseases, and strongly indicated the occurrence of their relations. In wet season, incidence of bacterial leaf streak, the damages by whorl maggots, the incidence of silver shoot and the number of brown plant hopper showed strong co-occurrence with other pests, and the incidence of narrow brown spot, the number of brown plant hopper, and white backed plant hopper showed strong co-occurrence in dry season. The strong co-occurrence of selected pests potentially indicated the key pests to control. 
\end{document}
















