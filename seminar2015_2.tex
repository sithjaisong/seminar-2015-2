\documentclass[12pt,a4paper]{article}
\usepackage{graphicx}
\usepackage[applemac]{inputenc} %% european characters can be used (Mac OS)
% ------------------------------------------------
\usepackage[T1]{fontenc}   %% get hyphenation and accented letters right
\usepackage{mathptmx}      %% use fitting times fonts also in formulas
\pagestyle{empty}                %% no page numbers!
\usepackage[left=35mm, right=35mm, top=15mm, bottom=20mm, noheadfoot]{geometry}

% begin the document
\begin{document}
\thispagestyle{empty}

\title{\textbf{Co-occurence Relationships Between Insect Pest and Disease from Farmer's Field Survey Data Revealed by Network Analysis}}
\author{Sith Jaisong \\
Plant Disease Management Group, CESD, IRRI\\ Los Ba\~{n}os, Philippines\\
s.jaisong@irri.org}
\date{} % <--- leave date empty
\maketitle\thispagestyle{empty} %% <-- you need this for the first page
% introduction should have the objective or the problem or even telling the reader what you wanna say, and 

% I'd like you to follow the American Phytopathological Society guidelines for abstract submission

%Presentation Title
%Capitalize only the first letter of the first word and any proper nouns, (e.g., Effect of pesticides on recovery of Didymella bryoniae from cucurbit vines). The title is limited to 150 characters including spaces. (Approximately 30 word count.) Registered names and trademarks are not permitted in title.
%
%Sith: I totally forget that you used to mention  the abstract should be following the role from American Phytopathology Society
%
%Abstract Text
%Read the technical requirements and view the sample abstract before submitting your abstract.
%
%The abstract must be in one paragraph.
%DO NOT include the title, author name(s), or author affiliations in the abstract text field.
%Copy the abstract and paste it into the submission form abstract text box under the Abstract Copy/Body field header.
%Or type text in to the abstract field.
%Use the abstract toolbar to add formatting (italics, superscripts, subscripts, Greek or math symbols), or use the start coding <i> and the stop coding </i>, if you prefer.
%If the symbol is not available, spell it out (e.g., theta).
%Character limit is 1,490 characters including spaces (Approximately 250 word count).
%


Pests are the blocks in the big wall of yield losses that agronomists have to deal with because of the need to increase crop productivity. Survey data from farmers' fields are useful sources of data to help us determining the importance of pests and understand the complex relationships of the agroecosystem. Co-occurrence patterns are used for explore potential relationships between elements within group of studies. Spearman's rank correlation-based network analysis was used to identify the co-occurrence correlations of the incidence of insect pests and diseases from survey data collected in 450 farmers' fields in irrigated lowland rice growing areas in five countries, India, Indonesia, Thailand and Vietnam from 2007 to 2010. Network models revealed relationships between incidences and severities of damages by insect pests and diseases, which showed strongly both positive and negative relationships within entities of network models. The network illustrated that the important insect pests and disease in South and South East Asia are brown plant hopper, whorl maggots, bacterial leaf streak and brown spot, which are determined by degree of connectivity. Moreover, network structures changed with different seasons (wet and dry seasons). In wet season, incidence of bacterial leaf streak, damage by whorl maggots, incidence of silver shoot and the number of brown plant hoppers showed strong co-occurrence. While the incidence of narrow brown spot, number of brown plant hoppers, and white backed plant hoppers showed strong co-occurrence in dry season. The strong co-occurrence of selected pests potentially indicate the key pests to control. 
 
\end{document}
















